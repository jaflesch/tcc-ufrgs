% 
% exemplo genérico de uso da classe iiufrgs.cls
% $Id: iiufrgs.tex,v 1.1.1.1 2005/01/18 23:54:42 avila Exp $
% 
% This is an example file and is hereby explicitly put in the
% public domain.
% 
\documentclass[cic,tc]{iiufrgs}
% Para usar o modelo, deve-se informar o programa e o tipo de documento.
% Programas :
% * cic       -- Graduação em Ciência da Computação
% * ecp       -- Graduação em Ciência da Computação
% * ppgc      -- Programa de Pós Graduação em Computação
% * pgmigro   -- Programa de Pós Graduação em Microeletrônica
% 
% Tipos de Documento:
% * tc                -- Trabalhos de Conclusão (apenas cic e ecp)
% * diss ou mestrado  -- Dissertações de Mestrado (ppgc e pgmicro)
% * tese ou doutorado -- Teses de Doutorado (ppgc e pgmicro)
% * ti                -- Trabalho Individual (ppgc e pgmicro)
% 
% Outras Opções:
% * english    -- para textos em inglês
% * openright  -- Força início de capítulos em páginas ímpares (padrão da
% biblioteca)
% * oneside    -- Desliga frente-e-verso
% * nominatalocal -- Lê os dados da nominata do arquivo nominatalocal.def

% === PACKAGES ===
% Use unicode
\usepackage[utf8]{inputenc}   % pacote para acentuação
% Necessário para incluir figuras
\usepackage{graphicx}         % pacote para importar figuras
\usepackage{times}            % pacote para usar fonte Adobe Times
% \usepackage{palatino}
% \usepackage{mathptmx}       % p/ usar fonte Adobe Times nas fórmulas
\usepackage[alf,abnt-emphasize=bf, abnt-etal-list=0, abnt-etal-cite=2]{abntex2cite}	% pacote para usar citações abnt
\usepackage{verbatim}
\usepackage{amsmath}
\usepackage{algorithm}
\usepackage[noend]{algpseudocode}
\usepackage{color, colortbl}
\usepackage{placeins}
\usepackage{chngpage}
\usepackage{hyperref}
\usepackage{rotating}
\usepackage{graphics}
\usepackage{array}
\usepackage{epigraph}
\usepackage{pdfpages}
\usepackage{lscape}


\usepackage{rotating} % Provides {sideways}{sidewaysfigure}{sidewaystable} environments


% === DEFINITIONS === 
\floatname{algorithm}{Algoritmo}
\graphicspath{ {./figuras/} }
\renewcommand{\arraystretch}{1.5}
\definecolor{darkorange}{rgb}{0.9647058823529412, 0.6980392156862745, 0.4196078431372549}
\definecolor{lightorange}{rgb}{0.9768, 0.79608, 0.61176}
\newcolumntype{o}{>{\columncolor{lightorange}}r}
\newcolumntype{P}[1]{>{\centering\arraybackslash}p{#1}}

% comandos miraculosos para evitar quebra de palavras
\tolerance=1
\emergencystretch=\maxdimen
\hyphenpenalty=10000
\hbadness=10000