\chapter{Conclusões}
\label{conclusao}

Neste trabalho foi apresentado o Portal de Vagas, um protótipo de rede social desenvolvido para explorar novas alternativas para a divulgação de bolsas e estágios na universidade. Além disso, a plataforma também simulou a integração com outras ferramentas já utilizadas pela UFRGS, como o Moodle, o Portal do Aluno e o Webmail em um ambiente de dados centralizado.

O trabalho começou com um estudo de trabalhos relacionados, em especial o Mural de Bolsas, onde foram detectadas funcionalidades essenciais para a rede social, que serviram de base para a análise de requisitos. Posteriormente, foi definida a metodologia ágil para o desenvolvimento da aplicação, dividida em quatro sprints baseadas em critérios de prioridade. 

A primeira entrega priorizou a funcionalidade de login e a construção das páginas básicas da rede social. A segunda, focou nas interações entre usuários como curtir, postar, seguir e bloquear e filtros padrões na busca de vagas. A terceira, finalizou as demais interações de usuário e complementou os filtros de busca. A quarta e última etapa, adicionou responsividade ao visual do sistema, aumentando a experiência de usuários provenientes de \textit{smartphones} e \textit{tablets}; ainda, otimizações de consulta ao banco e documentação apropriado para trabalhos futuros na plataforma.

Ao final das etapas de desenvolvimento, foi elaborado um experimento com usuários. Este consistia em um teste de usabilidade que comparou a realização de uma sequência de atividades pré-determinadas de usuário entre as ferramentas do Portal de Vagas e do Mural de Bolsas. A análise dos resultados obtidos com o experimento trouxe resultados interessantes.

<… complementar após os testes… >


Para trabalhos futuros, a primeira tarefa a ser realizada é a integração da ferramenta com as outras plataformas já utilizadas na UFRGS. Dessa forma, todos os usuários poderão usufruir das possibilidades ofertadas neste protótipo e em diversas outras que seriam criadas. Por exemplo, a integração com o Portal do Aluno poderia enviar o histórico escolar de um candidato ao autor de uma vaga durante sua candidatura. Outro cenário interessante seria o filtro de vagas incluir uma lista das disciplinas que o aluno está matriculado no Moodle e relacioná-las com os requisitos exigidos por cada vaga. Além disso, sugere-se também a inclusão de funcionalidades comuns em redes sociais que foram preteridas neste protótipo como compartilhar um post ou uma vaga, curtir comentários e recomendações.