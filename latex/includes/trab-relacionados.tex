\chapter{Trabalhos relacionados}

Neste capítulo são apresentados quatro sistemas de divulgação de vagas online em bolsas, estágios e empregos. O primeiro é a Plataforma de Gestão e Disseminação de Estágios Profissionais, uma aplicação Web para procura e divulgação de estágios profissionais entre alunos, diretores de cursos e empresas. O segundo é o Sistema de Gestão de Estágios e Empregos Online, que permite aos candidatos encontrar estágios e empregos através de processo de análise e seleção de currículos. O terceiro é a plataforma LinkedIn, uma rede social utilizada por profissionais para apresentarem suas aptidões. O quarto é o Mural de Bolsas, uma plataforma Web que facilita o preenchimento de vagas nas bolsas da UFRGS. Por fim, é realizada uma avaliação comparativa entre os quatro sistemas envolvendo critérios de funcionalidade e usabilidade.

\section{Plataforma de gestão e disseminação de estágios profissionais}
\label{trabRelPlatGestao}

A PGDEP \cite{PGDEPMono} foi implementada na Faculdade de Engenharia da Universidade do Porto, em Portugal, como proposta para melhorar o gerenciamento e indicação de estágios profissionalizantes mais adequados ao perfil dos alunos de cursos profissionalizantes. 

Entre as principais funcionalidades da plataforma estão:
\begin{itemize}
    \item Login no sistema
    \item Gerenciar alunos candidatos, diretor e coordenador de curso por escola
    \item Gerenciar empresas 
    \item Gerenciar cursos
    \item Alunos podem pesquisar e se candidatarem a estágios
    \item Coordenador do curso cadastra Diretor do curso e cursos profissionais
    \item Diretor do curso lista alunos para estágio e monitora estágios ofertados pelas empresas
    \item Galeria de trabalhos realizados por alunos nos cursos
    \item Registro de competências adquiridas pelos alunos em estágios
\end{itemize}

\begin{figure}[h]
    \caption{Tela inicial do PGDEP.}
       	\begin{center}
            \includegraphics[width=0.75\textwidth]{figuras/rel03.png}
        \end{center}
    \label{telaHomePGDEP}
    \legend{Fonte: \cite{PGDEPMono}}
\end{figure}

\begin{figure}[h]
    \caption{Página de gestão das escolas cadastradas na plataforma PGDEP.}
       	\begin{center}
            \includegraphics[width=0.75\textwidth]{figuras/rel04.png}
        \end{center}
    \label{telaEscolaPGDEP}
    \legend{Fonte: \cite{PGDEPMono}}
\end{figure}

\section{Sistema de gestão de estágios e empregos online}
\label{trabRelSistEmprego}

O SGEEO \cite{SGEEOMono} , é um sistema desenvolvido na Universidade de Mindelo, no Cabo Verde, que objetiva aumentar o controle e a participação da Universidade com seus alunos no processo de pesquisa e seleção de estágios e empregos. A plataforma \textit{online} possibilita aos alunos cadastrarem seus currículos, selecionando sua área de interesse. As empresas parceiras divulgam gratuitamente vagas de estágio ou emprego no sistema e os alunos interessados podem se inscrever e concorrer à vaga.

\begin{itemize}
    \item Entre as principais funcionalidades da plataforma estão:
    \item Login no sistema
    \item Gerenciar candidatos e parceiros
    \item Candidatos podem enviar currículo, selecionar áreas de interesse e pesquisar vagas
    \item Parceiros podem pesquisar candidatos a partir da lista de currículos cadastrados no sistema
    \item Chat global para conversa entre usuários
\end{itemize}

\begin{figure}[h]
    \caption{Tela inicial do SGEEO.}
       	\begin{center}
            \includegraphics[width=0.75\textwidth]{figuras/rel01.png}
        \end{center}
    \label{telaHomeSGEEO}
    \legend{Fonte: \cite{SGEEOMono}}
\end{figure}

\begin{figure}[h]
    \caption{Tela de cadastro do currículo do candidato no SGEEO.}
       	\begin{center}
            \includegraphics[width=0.75\textwidth]{figuras/rel02.png}
        \end{center}
    \label{telaCandidatoSGEEO}
    \legend{Fonte: \cite{SGEEOMono}}
\end{figure}

\section{LinkedIn}
\label{trabRelLinkedin}

LinkedIn\footnote{{\url{https://www.linkedin.com} Acesso em novembro de 2018}} é uma rede social voltada para negócios com grande destaque para o empreendedorismo e profissionais em geral. Através dele, usuários apresentam suas aptidões que são fortalecidas por recomendações de outros participantes da rede, promovendo interação e \textit{networking}. Além disso, é uma excelente plataforma para divulgação e pesquisa de vagas para as mais diversas categorias, incorporando desde estágios a programas de \textit{trainee} e, obviamente, de efetivo. A plataforma atualmente é a redes social de profissionais mais acessada e utilizada no mundo, contando com mais de 590 milhões de usuários\footnote{{\url{https://news.linkedin.com/about-us\#statistics} Acesso em novembro de 2018}}, sendo uma referência em termos de interação entre usuários, empresas e consequentes ofertas de emprego.

As principais funcionalidades da rede social são:
\begin{itemize}
    \item Login no sistema
    \item Curtir, comentar e compartilhar uma postagem
    \item Criar conexão com outro usuário
    \item Chat com usuários
    \item Pesquisar usuário e vagas
    \item Seguir usuários
    \item Recomendar usuário
    \item Criar artigos
    \item Ingressar em grupos de usuários
    \item Pesquisar, salvar, anunciar e compartilhar vagas
    \item Candidatar-se a vaga
\end{itemize}

\begin{figure}[h]
    \caption{Tela inicial de usuário no LinkedIn: \textit{feed} dos pessoas e páginas que o usuário segue.}
       	\begin{center}
            \includegraphics[width=0.85\textwidth]{figuras/rel05.png}
        \end{center}
    \label{telaHomeLKDIN}
    \legend{Fonte: LinkedIn}
\end{figure}

\section{Mural de Bolsas}
\label{trabRelMDB}

O MB\footnote{{\url{https://www.ufrgs.br/bolsas/}, Acesso em novembro de 2018}} é um portal online disponível para uso na UFRGS. Desenvolvido pela Empresa Jr. IDE\footnote{{\url{https://idejr.com.br/}, Acesso em novembro de 2018}}, composta por alunos de graduação de diferentes cursos e áreas de conhecimento na universidade. A proposta da plataforma foi reformar o antigo sistema da própria universidade que simplesmente divulgada uma tabela simples com poucas informações sobre as vagas disponíveis em bolsas. 

Com a reestruturação do site e, consequente construção do portal, experiência de usuário aumentou drasticamente  com a inclusão de várias funcionalidades. Entre elas, estão:
\begin{itemize}
    \item Login no sistema integrado com banco de dados da UFRGS
    \item Alunos podem pesquisar vagas
    \item Administradores podem gerenciar vagas
    \item Enviar e-mail com atualizações periódicas
    \item Configurar opções da conta de usuário
\end{itemize}

\begin{figure}[h]
    \caption{Página inicial do MB com informações do portal para alunos e professores.}
       	\begin{center}
            \includegraphics[width=0.68\textwidth]{figuras/rel06.png}
        \end{center}
    \label{telaHomeMB}
    \legend{Fonte: Empresa Jr. IDE}
\end{figure}

\begin{figure}[h]
    \caption{Tela de exibição das vagas disponíveis do portal do MB para alunos pesquisarem.}
       	\begin{center}
            \includegraphics[width=0.68\textwidth]{figuras/rel07.png}
        \end{center}
    \label{telaVagasMB}
    \legend{Fonte: Empresa Jr. IDE}
\end{figure}

\section{Análise comparativa}
\label{trabRelAnalise}

Através da análise das soluções das seções prévias, percebe-se muitas funcionalidades semelhantes e outras que divergem por apresentarem propostas e regras de negócio um diferentes. Com o objetivo de agrupar as principais características das plataformas e compará-las, será apresentada uma tabela com as principais características que se acredita ser um diferencial em sistemas de recomendação de vagas. 

\begin{table}[h]
    \begin{adjustwidth}{-0.7in}{-0.7in}
    \begin{center}
    \caption{Tabela comparativa sobre os trabalhos relacionados}
    \begin{tabular}{lllll}
    \hline
                        & PDGEP           & SGEEO             & LKDIN         & MB \\
    \hline
    \textit{Software}   & Sistema         & Sistema           & Rede social   & Sistema \\
    Oferta              & Estágios        & Estágios, empregos &Estágios, \textit{trainees}, empregos & Bolsas \\
    Login               & Sim             & Sim               & Sim           & Sim \\
    Enviar currículo    & Não             & Sim               & Sim           & Não \\
    Pesquisar vagas     & Sim             & Sim               & Sim           & Sim \\
    Pesquisar usuários  & Apenas empresas & Apenas parceiros  & Sim           & Não \\
    Candidatar-se a vaga& Sim             & Sim               & Sim           & Não \\
    \textit{Feedback} do candidato & Sim  & Não               & Sim           & Não \\
    Preferências do usuário & Sim         & Sim               & Sim           & Sim \\
    Recomendações       & Sim             & Não               & Sim           & Não \\
    Favoritar vagas     & Não             & Não               & Sim           & Não \\
    Seguir usuário      & Não             & Não               & Sim           & Não \\
    Amizades            & Não             & Não               & Sim (conexões) & Não \\
    Fazer posts         & Não             & Não               & Sim           & Não \\
    Fazer comentários   & Não             & Não               & Sim           & Não \\
    Compartilhar vagas  & Não             & Não               & Sim           & Não \\
    Grupos de usuários  & Não             & Não               & Sim           & Não \\
    Chat Online         & Não             & Global            & Individual    & Não \\
    \hline
    
    \end{tabular}
    \end{center}
    \end{adjustwidth}
    \bigskip
    \legend{Tabela com as principais funcionalidades dos quatro sistemas apresentados neste capítulo. A sigla LKDIN significa LinkedIn. }
    \label{tabelaAvalTrab}
\end{table}