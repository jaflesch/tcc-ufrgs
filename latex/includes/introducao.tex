\chapter{Introdução}

\section{Motivação}
\label{introducaoMotivacao}

Vivemos em uma época onde a informação está ao nosso alcance e muitas vezes não sabemos como utilizar todas as possibilidades que as tecnologias nos oferecem \cite{anotherNodeArticle} \cite{impactInternetArticle}. Na universidade, não é incomum conhecer alunos que desejam colocar em prática os conhecimentos adquiridos ao longo das disciplinas cursadas na graduação, mas que apresentam dificuldades para encontrar alguma bolsa -seja de iniciação científica ou não-,
estágio ou até vagas no mercado de trabalho para suas áreas de interesse \cite{internetLibDesArticle}. Para tal, possibilitar ferramentas e recursos para que alunos consigam trabalhar desenvolvendo e/ou pesquisando é fundamental para o crescimento profissional e acadêmico de qualquer graduando \cite{teachMediaArticle}. Entretanto, às vezes, mesmo com o alcance da informação, não existe uma organização nem um local onde todos os conteúdos relevantes para um nicho específico de usuários se concentre, fazendo com que sejam necessárias buscas exaustivas navegando até locais de interesse, geralmente escondidos e, consequentemente, levando-nos ao problema inicialmente abordado \cite{socConnectArticle}.

Atualmente, a única ferramenta que se aproxima da solução ideal na UFRGS é o Mural de Bolsas\footnote{A ferramenta está disponível em \url{https://www.ufrgs.br/bolsas/}. Para ter acesso à lista de vagas, é necessário realizar \textit{login} através das credenciais utilizadas no site da UFRGS.}, que passou por uma releitura recente. O Mural de Bolsas apenas lista e possibilita alguns filtros de pesquisa para os alunos interessados em vagas ofertadas de bolsas. Não existe comunicação direta na plataforma e as interações se resumem apenas a pesquisar opções de vagas disponíveis. Não há registro nem histórico de modo organizado e intuitivo das ações de alunos (pelo menos que eles possam acompanhar), tampouco resolve o problema de múltiplas informações descentralizadas \cite{socialChallengeArticle}.

A proposta deste trabalho é a criação de uma plataforma que agrupe diversas ferramentas que a UFRGS já oferece aos seus alunos, como o Moodle e o Portal do Aluno, por exemplo, e realize uma integração entre estas, de forma a centralizar todas as informações úteis e relevantes sobre os alunos em um único ambiente \cite{socConnectArticle}, tornando possível pesquisar vagas de bolsas e também realizar interações entre alunos e professores para que possam conversar, trocar ideias e recomendar uns aos outros através de buscas mais refinadas \cite{UXLinkedinArticle} \cite{agileCareerArticle}. 

Um cenário interessante e que ainda não é possível de ser feito com os recursos atuais é a possibilidade de ofertar uma bolsa para todos os alunos matriculados em uma disciplina, àqueles que dominem determinada tecnologia, que possuam um histórico escolar qualificado entre outros \cite{designCurriculumArticle} \cite{minigLinkedinInbook}. Dessa forma, as bolsas podem ser preenchidas por alunos mais
qualificados e que realmente estejam dispostos a aprender, pois uma má conduta (que infelizmente acontece) ficaria registrada no sistema e os próximos professores poderiam se valer dessa informação antes de oferecem uma nova bolsa ao mesmo aluno \cite{academicAchievmentArticle}, evitando que
outro colega mais determinado perca essa oportunidade \cite{futureITLinkedinArticle}. 

Além disso, por ser uma rede social, outra grande vantagem é o estabelecimento de \textit{networkings}. Alunos e professores podem manter contato mesmo após a graduação e também ajudar o aluno caso este planeje um mestrado e/ou doutorado \cite{alumniInbook}. Alunos também podem interagir entre si, divulgando vagas de emprego onde trabalham e recomendando outros colegas \cite{recommendPeopleInbook}, caso optem seguir um lado mais empreendedor \cite{SNsEntrepreunersInbook} ou fora do meio acadêmico \cite{globalParticipationInbook}.


\section{Objetivos}
\label{introducaoObjetivos}
Este trabalho foi produzido com a proposta de explorar, através da implementação de um protótipo de rede social, novas alternativas para a divulgação de vagas em bolsas e estágios ofertados pela UFRGS. Complementarmente, foram propostos dois experimentos neste trabalho: um teste A/B \cite{abTestArticle} para comparar características mútuas entre o protótipo desenvolvido com a plataforma Mural de Bolsas da UFRGS e um teste de usabilidade para voluntários validarem funcionalidades implementadas neste protótipo. Acredita-se que com uma ferramenta capaz de promover interação entre os usuários, que ofereça critérios de busca e classificação refinados de candidatos a vagas e, ainda, seja integrada com outras plataformas da própria universidade, acarrete em um aumento da atratividade e do público-alvo do sistema final.


\section{Organização do texto}
\label{introducaoOrganizacao}
O restante do texto apresenta-se dividido em 6 capítulos. O Capítulo 2, de fundamentação teórica, apresenta as definições relevantes ao desenvolvimento deste trabalho. O Capítulo 3 abrange trabalhos relacionados disponíveis academicamente e no mercado. O Capítulo 4 apresenta a metodologia de desenvolvimento utilizada para a implementação da solução proposta. O Capítulo 5 explora as principais funcionalidades da rede social Portal de Vagas. O Capítulo 6 descreve o experimento realizado com os usuários e apresenta os resultados obtidos, assim como uma análise destes. No Capítulo 7, por fim, é realizada uma síntese do trabalho desenvolvido, elencando o que foi alcançado e quais direções para trabalhos futuros.