% Resumo
\begin{abstract}
	É durante a graduação que geralmente ocorrem as primeiras experiências profissionais dos alunos, sejam elas provenientes de bolsas ofertadas pela própria universidade ou de estágios em empresas do mercado de trabalho. Para ajudar na divulgação de vagas em ambas as modalidades, a UFRGS disponibiliza ferramentas de acesso à informação aos seus discentes como o \textit{e-mail} de graduação, Twitter e o site do Mural de Bolsas. Este  trabalho  objetiva realizar  uma  releitura  do atual Mural  de  Bolsas, desenvolvendo  uma rede social que aproxime ainda mais alunos e professores, possibilite interações mais dinâmicas e forneça um canal de comunicação para os usuários, tudo isto centralizado em uma única ferramenta. Desta forma, é possível desenvolver soluções mais eficientes e precisas no gerenciamento das vagas ofertadas, utilizando pesquisas mais refinadas e complexas como as baseadas no desempenho acadêmico do aluno, por exemplo. Adicionalmente, foram aplicados experimentos a voluntários para avaliar a ferramenta proposta. Os resultados indicam uma forte aceitação às mudanças e funcionalidades implementadas, além de elucidar características importantes para os usuários em aplicações deste tipo. Acredita-se que a rede social contribuirá para uma maior organização na divulgação de bolsas e estágios na universidade e um consequente ingresso mais qualificado de alunos, privilegiando àqueles com real interesse nas vagas e melhor desempenho em seu curso de graduação.
    \newline
\end{abstract}

% Abstract
% como parametros devem ser passados o titulo e as palavras-chave
% na outra língua, separadas por vírgulas
\begin{englishabstract}{
Portal de Vagas - implementation of a tool to promote scholarships and internships to students}{Social network.  Web application. Professional experience. User Experience}
	It is during the graduation that usually the first professional experiences of the students occur, whether they come from scholarships offered by the university itself or from internships in companies of the labor market. To help publicize job opportunity in both modalities, UFRGS provides tools for accessing information to its students, such as the graduation e-mail and the \textit{Mural de Bolsas} website. This work aims to re-read the current \textit{Mural de Bolsas}, developing a social network that brings even more students and teachers, enables more dynamic interactions and provides a channel of communication for users, all centralized in a single platform. In this way, it is possible to develop more efficient and precise solutions in the management of offered job opportunities, using more refined and complex researches such as those based on the academic performance of the student, for example. In addition, experiments were applied to volunteers to evaluate the proposed tool. The results indicate a strong acceptance to the changes and functionalities implemented, as well as to elucidate important characteristics for the users in applications of this type. It is believed that the social network will contribute to a greater organization in the dissemination of scholarships and internships in the university and a consequent more qualified admission of students, privileging those with real interest in the job opportunities and better performance in its undergraduate course.

\end{englishabstract}